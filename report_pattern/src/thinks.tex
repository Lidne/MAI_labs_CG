\section{Результаты}

В ходе выполнения лабораторной работы были изучены основы 3D-графики и принципы построения изображений с использованием графического API Vulkan.

Были реализованы и применены на практике:
\begin{itemize}
    \item Математический аппарат для работы с матрицами 4x4 (проекция, вращение, перемещение).
    \item Базовая настройка графического конвейера Vulkan.
    \item Передача данных (вершин, матриц) в шейдеры.
\end{itemize}

Результатом работы является программа, отображающая 3D-куб с возможностью управления его вращением и положением.

\begin{figure}[h]
    \centering
    \includegraphics[width=0.8\textwidth]{images/image.png}
    \caption{Демонстрация работы программы}
    \label{fig:screenshot}
\end{figure}

В ходе выполнения лабораторной работы были успешно освоены ключевые аспекты работы с 3D-графикой и Vulkan API. Реализованная программа демонстрирует корректную работу с матричными преобразованиями, что является основой для дальнейшего изучения компьютерной графики. Применение проекционных матриц позволило получить реалистичное отображение трехмерных объектов на двумерном экране. Работа с шейдерами и передача данных через uniform-буферы показала практическое применение современных графических технологий.

\pagebreak