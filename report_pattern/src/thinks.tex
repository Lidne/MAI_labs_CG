\section{Выводы}
В ходе выполнения лабораторной работы была решена задача о подсчете количества отрезков, лежащих строго над заданными точками. Для решения была выбрана структура данных \textbf{персистентное дерево отрезков} в сочетании с методом \textbf{сжатия координат}.

Основные выводы:
\begin{itemize}
    \item Использование персистентности позволяет эффективно решать задачи, содержащие временное измерение или измерение, которое можно свести к временному (в данном случае — высота $h$).
    \item Сжатие координат является необходимым этапом при работе с большими координатами ($10^9$) в задачах на деревьях отрезков, позволяя ограничить размер дерева количеством "интересных" точек ($O(N+M)$).
    \item Подход с разностным массивом на дереве отрезков (прибавление $+1$ и $-1$) позволяет свести задачу проверки вхождения в отрезок к задаче вычисления префиксной суммы.
\end{itemize}

Решение полностью поддерживает online-запросы, так как структура данных строится один раз по известным отрезкам, а запросы точек обрабатываются независимо.

\pagebreak
