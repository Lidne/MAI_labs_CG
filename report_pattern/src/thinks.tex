\section{Результаты}

В ходе выполнения лабораторной работы были изучены механизмы работы с текстурами в графическом API Vulkan.

Были реализованы следующие этапы:
\begin{itemize}
    \item Загрузка изображения формата PNG с диска в оперативную память.
    \item Создание объектов Vulkan: \texttt{VkImage}, \texttt{VkImageView}, \texttt{VkSampler}.
    \item Настройка барьеров памяти для корректного перевода изображения в оптимальный для чтения шейдером формат (\texttt{VK\_IMAGE\_LAYOUT\_SHADER\_READ\_ONLY\_OPTIMAL}).
    \item Модификация графического конвейера для передачи текстурных координат и сэмплеров в шейдеры.
\end{itemize}

В результате на 3D-объекты сцены (кубы и плоскость) корректно накладывается текстура. Текстура корректно взаимодействует с моделью освещения, модулируя базовый цвет объектов.

\begin{figure}[h]
    \centering
    \includegraphics[width=0.8\textwidth]{images/image.png}
    \caption{Результат применения текстуры к объектам сцены}
\end{figure}

В ходе выполнения лабораторной работы был успешно реализован механизм работы с текстурами в Vulkan. Изучены основные концепции работы с изображениями в графическом API: создание объектов изображений, управление памятью, настройка сэмплеров и передача данных в шейдеры. Полученные знания позволяют создавать визуально более привлекательные 3D-сцены с использованием текстур, что является важным шагом в освоении современной графической разработки.

\pagebreak
