\section{Результаты}

В ходе выполнения лабораторной работы была успешно реализована техника Shadow Mapping.

Основные достижения:
\begin{itemize}
    \item Реализован рендеринг сцены в текстуру глубины с позиции источника света (Off-screen rendering).
    \item Настроен графический конвейер Vulkan для генерации карты теней.
    \item Внедрена техника PCF (Percentage Closer Filtering) для сглаживания краев теней.
    \item Реализован механизм Shadow Bias для устранения визуальных артефактов самозатенения.
    \item Тени корректно накладываются на объекты сцены, учитывая их взаимное расположение и направление света.
\end{itemize}

Полученная программа позволяет визуализировать 3D-сцену с динамическими тенями от направленного источника света, что значительно повышает реалистичность изображения.

\begin{figure}[h]
    \centering
    \includegraphics[width=0.8\textwidth]{images/image1.png}
    \caption{Демонстрация работы программы: наложение теней}
    \label{fig:screenshot_shadows1}
\end{figure}

\begin{figure}[h]
    \centering
    \includegraphics[width=0.8\textwidth]{images/image2.png}
    \caption{Демонстрация работы программы: визуализация теней}
    \label{fig:screenshot_shadows2}
\end{figure}

\subsection{Выводы}

В результате выполнения лабораторной работы была успешно реализована техника Shadow Mapping для создания реалистичных теней в 3D-сцене. Реализованное решение демонстрирует корректную работу алгоритма: тени правильно накладываются на объекты, учитывая их взаимное расположение и направление света. Применение техники PCF позволило получить мягкие края теней, что улучшает визуальное качество изображения. Использование Shadow Bias устранило артефакты самозатенения, обеспечив стабильную работу алгоритма.

Реализованная система может быть использована для создания более реалистичных визуализаций в приложениях компьютерной графики, работающих на платформе Vulkan.

\pagebreak
