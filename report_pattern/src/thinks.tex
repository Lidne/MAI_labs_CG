\section{Результаты}

В ходе выполнения лабораторной работы были реализованы:

\begin{enumerate}
    \item \textbf{Свободная камера:}
    \begin{itemize}
        \item Реализовано управление положением и ориентацией камеры с помощью клавиатуры и мыши.
        \item Камера корректно обрабатывает перспективную проекцию и перемещение в 3D-пространстве.
    \end{itemize}

    \item \textbf{Система освещения:}
    \begin{itemize}
        \item Реализована модель освещения Блинн-Фонга во фрагментном шейдере.
        \item Поддерживается одновременная работа направленного источника света и нескольких точечных источников.
        \item Реализован корректный расчет диффузной и спекулярной составляющих, а также затухания света для точечных источников.
    \end{itemize}
\end{enumerate}

Результатом работы является интерактивное 3D-приложение, позволяющее пользователю перемещаться по сцене и настраивать параметры освещения в реальном времени через графический интерфейс. Визуализация демонстрирует корректные блики (specular highlights) и затенение объектов.

\begin{figure}[h]
    \centering
    \includegraphics[width=0.8\textwidth]{images/image.png}
    \caption{Скриншот работы приложения с реализованной системой освещения}
\end{figure}

\begin{figure}[h]
    \centering
    \includegraphics[width=0.8\textwidth]{images/image1.png}
    \caption{Демонстрация различных параметров освещения и эффектов}
\end{figure}

\subsection{Выводы}

В ходе выполнения лабораторной работы была успешно реализована система свободной камеры и освещения на основе модели Блинн-Фонга. Приложение демонстрирует корректную работу всех компонентов: камера плавно перемещается в 3D-пространстве, система освещения правильно рассчитывает диффузную и спекулярную составляющие, а также корректно обрабатывает затухание света для точечных источников.

Реализованное решение позволяет интерактивно настраивать параметры освещения в реальном времени, что обеспечивает удобство тестирования и демонстрации различных визуальных эффектов. Приложение может быть использовано как основа для дальнейшей разработки более сложных графических приложений с расширенными возможностями рендеринга.

\pagebreak
