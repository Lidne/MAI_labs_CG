\CWHeader{Курсовая работа}

\CWProblem{
Вам дан набор горизонтальных отрезков и набор точек. Для каждой точки определите, сколько отрезков лежит строго над ней.

Ваше решение должно работать online, то есть должно обрабатывать запросы по одному после построения необходимой структуры данных по входным данным. Чтение входных данных и запросов вместе и построение по ним общей структуры запрещено.

Формат ввода:
В первой строке даны два числа $n$ и $m$ ($1 \le n, m \le 10^5$) — количество отрезков и точек.
В следующих $n$ строках заданы отрезки тройками $l, r, h$ ($-10^9 \le l < r \le 10^9$, $-10^9 \le h \le 10^9$).
В следующих $m$ строках даны точки парами $x, y$ ($-10^9 \le x, y \le 10^9$).

Формат вывода:
Для каждой точки выведите количество отрезков строго над ней.

Метод решения:
Персистентное дерево отрезков с сжатием координат.
}
\pagebreak
