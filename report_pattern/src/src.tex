\section{Метод решения}

Для реализации техники наложения теней (Shadow Mapping) были выполнены следующие шаги:

\subsection{Создание карты теней}
Была создана текстура глубины, в которую будет производиться рендеринг сцены с точки зрения источника света.
\begin{itemize}
    \item Формат изображения: \texttt{VK\_FORMAT\_D32\_SFLOAT}.
    \item Использование: \texttt{VK\_IMAGE\_USAGE\_DEPTH\_STENCIL\_ATTACHMENT\_BIT | VK\_IMAGE\_USAGE\_SAMPLED\_BIT}.
    \item Создан сэмплер с поддержкой сравнения глубины (\texttt{VK\_COMPARE\_OP\_LESS\_OR\_EQUAL}) для реализации PCF (Percentage Closer Filtering).
\end{itemize}

\subsection{Проход рендеринга теней (Shadow Pass)}
Реализован отдельный проход рендеринга (\texttt{VkRenderPass}) для заполнения карты теней.
\begin{itemize}
    \item Используется только attachment глубины, запись цвета отключена.
    \item Графический конвейер (\texttt{shadow\_pipeline}) настроен на использование только вершинного шейдера (\texttt{shadow.vert}) для трансформации вершин в пространство света.
    \item Включен \texttt{depthTestEnable} и \texttt{depthWriteEnable}.
    \item \texttt{cullMode} установлен в \texttt{VK\_CULL\_MODE\_FRONT\_BIT} (или настроен соответственно) для устранения артефактов "peter panning".
\end{itemize}

\subsection{Использование карты теней}
В основном проходе рендеринга карта теней передается во фрагментный шейдер (\texttt{shader.frag}) через дескриптор.
\begin{itemize}
    \item Вершинный шейдер вычисляет позицию фрагмента в пространстве света (\texttt{fragPosLightSpace}).
    \item Во фрагментном шейдере реализована функция \texttt{ShadowCalculation}:
    \begin{itemize}
        \item Выполняется перспективное деление.
        \item Применяется смещение (bias) для устранения артефактов "теневой ряби" (Shadow Acne), зависящее от угла падения света.
        \item Реализована фильтрация PCF: выборка производится из окрестности текущего фрагмента (ядро 3x3), результаты усредняются для получения мягких краев тени.
    \end{itemize}
    \item Итоговое значение освещенности (diffuse и specular) умножается на коэффициент \texttt{(1.0 - shadow)}.
\end{itemize}

\subsection{Интеграция}
Код рендеринга карты теней (\texttt{renderShadowMap}) выполняется перед основным рендерингом сцены в каждом кадре. Реализована корректная синхронизация ресурсов (Subpass Dependencies) для обеспечения завершения записи в карту теней перед её чтением в основном проходе.
