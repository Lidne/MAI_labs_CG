\section{Метод решения}

Для реализации поставленной задачи была разработана программа на языке C++ с использованием графического API Vulkan.

Основные математические операции были реализованы вручную в файле \texttt{testbed/main.cpp}:
\begin{itemize}
    \item Матрица проекции (\texttt{projection}) — формирует матрицу перспективной проекции на основе угла обзора (FOV), соотношения сторон и плоскостей отсечения.
    \item Аффинные преобразования — реализованы функции создания матриц трансляции (\texttt{translation}) и вращения (\texttt{rotation}) вокруг произвольной оси.
    \item Операции с матрицами — функция перемножения матриц (\texttt{multiply}) для комбинирования модельных преобразований.
\end{itemize}

Инициализация и работа с Vulkan (файл \texttt{source/veekay.cpp} и \texttt{testbed/main.cpp}):
\begin{itemize}
    \item Создание \texttt{VkInstance}, \texttt{VkDevice}, \texttt{VkSwapchainKHR} для взаимодействия с оконной системой (GLFW).
    \item Загрузка шейдеров (\texttt{VkShaderModule}) и создание графического конвейера (\texttt{VkPipeline}). Вершинный шейдер (\texttt{shader.vert}) принимает матрицы трансформации и проекции через Push Constants.
    \item Управление памятью и буферами (\texttt{VkBuffer}, \texttt{VkDeviceMemory}) для хранения вершинных и индексных данных 3D-модели (куба).
    \item Синхронизация кадров с использованием \texttt{VkSemaphore} и \texttt{VkFence}.
\end{itemize}

В качестве интерактивного интерфейса для изменения параметров трансформации в реальном времени была интегрирована библиотека ImGui.