\section{Метод решения}

Для реализации поставленной задачи была разработана программа на языке C++ с использованием графического API Vulkan.

\subsection{Реализация камеры}
Камера реализована в виде структуры \texttt{Camera}, хранящей:
\begin{itemize}
    \item Позицию (\texttt{vec3 position});
    \item Углы поворота (\texttt{vec3 rotation});
    \item Параметры проекции (FOV, ближняя и дальняя плоскости отсечения).
\end{itemize}

Формирование матрицы вида (\texttt{view matrix}) и матрицы проекции (\texttt{projection matrix}) происходит в методах \texttt{view()} и \texttt{view\_projection()}.

Управление камерой реализовано следующим образом:
\begin{itemize}
    \item \textbf{Вращение:} Используется мышь. Изменение координат курсора преобразуется в изменение углов \texttt{yaw} и \texttt{pitch}. Для предотвращения "кувырка" камеры угол тангажа (\texttt{pitch}) ограничен диапазоном $[-89^\circ, 89^\circ]$.
    \item \textbf{Перемещение:} Используются клавиши клавиатуры (WASD, Q, E). Векторы направления движения (вперед, вправо, вверх) извлекаются из текущей матрицы вида, что позволяет перемещаться относительно направления взгляда камеры.
\end{itemize}

\subsection{Реализация освещения (Blinn-Phong)}
Освещение реализовано во фрагментном шейдере (\texttt{shader.frag}) с использованием модели Блинн-Фонга. Эта модель является модификацией модели Фонга, где вместо вычисления вектора отражения используется "серединный вектор" (halfway vector).

$$ H = \frac{L + V}{||L + V||} $$
где $L$ — вектор на источник света, $V$ — вектор на наблюдателя (камеру).

Спекулярная составляющая вычисляется как скалярное произведение нормали поверхности $N$ и серединного вектора $H$, возведенное в степень блеска (shininess):
$$ I_{spec} = (N \cdot H)^{\textrm{shininess}} $$

В программе реализованы два типа источников света:
\begin{enumerate}
    \item \textbf{Направленный свет (Directional Light):} Имитирует удаленный источник (например, солнце). Лучи света параллельны, интенсивность не зависит от расстояния.
    \item \textbf{Точечный свет (Point Light):} Источник находится в конкретной точке пространства и излучает свет во всех направлениях. Реализовано затухание (attenuation) света в зависимости от расстояния $d$:
    $$ Att = \frac{1}{K_c + K_l \cdot d + K_q \cdot d^2} $$
    где $K_c, K_l, K_q$ — постоянный, линейный и квадратичный коэффициенты затухания.
\end{enumerate}

Данные об источниках света передаются в шейдер через Uniform-буферы. Для интерактивного изменения параметров освещения (цвет, положение, затухание) интегрирован интерфейс ImGui.
